\documentclass{llncs}

\usepackage{amssymb}

\begin{document}

\title{Poset Paper}
\author{Bow yow Wang \and Chih chen Yuan}
\institute{Institute of Information Science, Academia Sinica}
\maketitle

\begin{abstract}
hello world
\end{abstract}

\section{Introduction}
Intro part here

\section{Preliminaries}
A partial order is a binary relation $\leq$ that is reflexive, antisymmetric, and transitive. A set $\mathcal{S}$ with a partial order is called a partially ordered set(poset). A linear order $<$ is a partial order that is total. If $\leq = \emptyset$, we call $\leq$ the discreet order. A linear extension $<$ of a partial order $\leq$ is an extension of $\leq$ that is total. Let $\mathcal{E}(\leq)$ be set of all linear extensions of $\leq$. We wrtie $P \sqsubseteq L$ iff L extends P and L is a linear order.

For a poset $P$ and $x, y \in \mathcal{P}$, let it be that $x \prec_{P} y \triangleq x \leq_{P} y \wedge \not\exists z \in \mathcal{P}, x \leq_{P} z \wedge z \leq_{P} y$; $x \parallel_{P} y \triangleq x \not\leq_{P} y \wedge y \not\leq_{P} x$, and, for linear extension $L, L' \in \mathcal{E}(P)$, that $L \leftrightarrow_{x, y} L' \triangleq x \prec_{L} y \wedge y \prec_{L'} x$.

Let the swap graph $\mathcal{G}(P)$ of poset $P$ be the undirected graph $(V,E)$ such that $V \triangleq \mathcal{E}(\leq)$ and
$E \triangleq \{(L, L') | \exists x,y \in P, L \leftrightarrow_{x, y} L'\}$.

The Hasse diagram of a poset $P$ is a directed acyclic graph constructed with transitivity removed. If $L$ is a linear order, we shall consider it in string form of linearizations. For a poset $P$ with $\leq$ let $\mathcal{L}(P)$ be $\mathcal{E}(\leq)$ in string form; then, it is a subset of the permutations of elements of $P$; that is, $\mathcal{L}(P) \subseteq \mathfrak{S}(\mathcal{P})$. Note that in the case of discreet order, $\mathcal{L}(P) = \mathfrak{S}(\mathcal{P})$, and that for any poset, $\mathcal{L}(P)$ is the set of topological sort, in string form, of the Hasse diagram.

The poset cover problem finds, given a set of linearizations $\Upsilon$, a set of posets $\mathcal{C}$, called a cover, such that $\Upsilon = \bigcup_{P \in C} \mathcal{L}(P)$ and that $|\mathcal{C}|$ is minimal. This problem is proved to be NP-complete.(cite)

\section{Related Theorems}

\begin{theorem}
    Permutations and Nerode
\end{theorem}

\begin{theorem}
    (Corollary of Szpilrajn extension theorem) For a partial order $\leq$, $\leq = \bigcap_{< \in \mathcal{E}(\leq)} <$.
\end{theorem}

\begin{theorem}
    (Heath and Nema) If $x \parallel y$ for $\leq$ and there is $< \in \mathcal{E}(\leq)$ such that $x \prec y$ for $<$, then there is $<' \in \mathcal{E}(\leq)$ such that $y \prec x$ for $<'$.
\end{theorem}

\begin{theorem}
    (Pruesse? and Ruskey) Swap graphs are connected
\end{theorem}

\begin{definition}
    (Poset cover problem) A poset cover $\mathcal{C}$ of a set posets on $S$ for a set of linearizations $\Upsilon$ satisfies the following: (1) (extension constraint) $\forall L \in \Upsilon \exists P \in \mathcal{C}, P \sqsubseteq L$; (2) (non-extension constraint) $\forall L \in \mathfrak{S}(\mathcal{S}) - \Upsilon \forall P \in \mathcal{C}, P \not\sqsubseteq L$.
\end{definition}

\begin{theorem}
    insulating barrier method
\end{theorem}

\begin{theorem}
    Swap graphs are connected by Kendall tau paths
\end{theorem}

\section{SAT Encoding Part}
We use the boolean logic of z3 SMT solver from MS research. Since poset is connected, by contraposition, we can first divide and conquer on connected components of the swap graph, by solving each component individually.

We then check with incrementing number of posets to find the minimal number of posets to cover each component. We first encode the axioms of posets; namely, reflexivity, antisymmetry, and transitivity. Next, we encode the extension constraint with contraposition from input linearizations to reduce the number of variables. Finally, we encode the non-extension constraint with negation of extension constraint on the insulating barrier.

\section{Exp Part}
NOTE: how do i put graph here?

\section{Conclusions}

\section{References}
NOTE: how to use list?

\end{document}
