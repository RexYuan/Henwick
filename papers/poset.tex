\documentclass{llncs}

\usepackage{amssymb}

\begin{document}

\title{Poset Paper}
\author{Bow-yaw Wang \and Chih-chen Yuan \and someone}
\institute{Institute of Information Science\\
Academia Sinica\\
\email{\{bywang,rex\}@iis.sinica.edu}}
\maketitle

\begin{abstract}
hello world
\end{abstract}

\section{Introduction}
Intro part here

\section{Preliminaries}

A partial order is a binary relation that is reflexive, antisymmetric, and transitive. A set $\mathrm{\Omega}$ equipped with a partial order $\leq$ is called a partially ordered set or poset, described in terms of a binary structure $\mathcal{P} = (\mathrm{\Omega}, \leq)$; we refer to members of $\mathrm{\Omega}$ as elements of $\mathcal{P}$ and, where specificity is desired, to $\mathrm{\Omega}$ as $\mathrm{\Omega}_{\mathcal{P}}$ and $\leq$ as $\leq_{\mathcal{P}}$.

\begin{example}
    For the following definition, consider the poset $\mathcal{P}$ over $\{a,b,c,d\}$ with $a \leq b$; $a \leq c$; $a \leq d$; $b \leq d$; and $c \leq d$.
\end{example}

The cover relation $\prec$ of a poset $\mathcal{P}$ is the transitive reduction of the order relation; it describes the case of immediate successor: for $x, y \in \mathcal{P}$, $x \prec y$ iff $x \leq y \wedge \not\exists z \in \mathcal{P}, x \leq z \wedge z \leq y$. For *Example 1*, the cover relation includes $a \prec b$; $a \prec c$; $b \prec d$; and $c \prec d$. Note that $(a, d)$ is absent in $\prec$.

In addition, we describe the case of incomparability in a poset $\mathcal{P}$ with $\parallel$: for $x, y \in \mathcal{P}$, $x \parallel y$ iff $x \not\leq y \wedge y \not\leq x$. For *Example 1*, there is $b \parallel c$.

A Hasse diagram $\mathcal{H}$ sums up a poset $\mathcal{P} = (\mathrm{\Omega}, \leq)$ with a directed acyclic graph $\mathcal{H} = (\mathrm{V},\mathrm{E})$, where the nodes $\mathrm{V} = \mathrm{\Omega}$ and the edges $\mathrm{E} = \prec$. *Figure 1* shows the Hasse diagram of *Example 1*.

\begin{example}
    For the following definition, consider the poset $\mathcal{L}$ over $\{a,b,c,d\}$ with $a \leq b$; $a \leq c$; $a \leq d$; $b \leq c$; $b \leq d$; and $c \leq d$. The Hasse diagram is shown in *Figure 2*.
\end{example}

A partial order that is also total is called a linear order; every pair of elements are comparable in a linear order. Note that the Hasse diagram of a linear order is a chain. For simplicity, we represent a linear order in string form. For *Example 2*, we write $abcd$.

\begin{theorem}
    (Szpilrajn Theorem) For a partial order $\mathcal{P}$, there exists a linear order $\mathcal{L}$ that extends $\mathcal{P}$; that is, $\leq_{\mathcal{P}} \subseteq \leq_{\mathcal{L}}$.
\end{theorem}

For a poset $\mathcal{P}$, a linear order $\mathcal{L}$ that extends $\mathcal{P}$ is called a linear extension or linearization of $\mathcal{P}$, and we write $\mathcal{P} \sqsubseteq \mathcal{L}$ iff $\mathcal{L}$ is a linear extension of $\mathcal{P}$. For example, for $\mathcal{P}$ from *Example 1* and $\mathcal{L}$ from *Example 2*, $\mathcal{P} \sqsubseteq \mathcal{L}$.

The set of all linearizations of $\mathcal{P}$ is denoted $\mathcal{L}(\mathcal{P})$. We shall consider it the language of a poset[Appendix]. By *Theorem 1*, $\mathcal{L}(\mathcal{P}) \neq \emptyset$. Note that a linearization of a poset is equivalent to a topological sort of the Hasse diagram of that poset. For $\mathcal{P}$ from *Example 1*, $\mathcal{L}(\mathcal{P}) = \{abcd, acbd\}$.

The swap relation $\leftrightarrow$ between linear orders with shared universe is defined as: for linear orders $\mathcal{L}_{1}, \mathcal{L}_{2}$ and $x, y \in \mathrm{\Omega}$, $\mathcal{L}_{1} \leftrightarrow_{x, y} \mathcal{L}_{2}$ iff $x \prec_{\mathcal{L}_{1}} y \wedge y \prec_{\mathcal{L}_{2}} x$. For example, $abcd \leftrightarrow_{b, c} acbd$. Note that $\leftrightarrow$ is symmetric.

The swap graph $\mathcal{G}(\mathcal{P})$ for a poset $\mathcal{P}$ is constructed as the undirected graph $(\mathrm{V},\mathrm{E})$ such that $\mathrm{V} = \mathcal{L}(\mathcal{P})$ and $\mathrm{E} = \{(\mathcal{L}_{1}, \mathcal{L}_{2}) | \exists x,y \in \mathcal{P}, \mathcal{L}_{1} \leftrightarrow_{x, y} \mathcal{L}_{2}\}$. *Figure 3* shows the swap graph $\mathcal{G}(\mathcal{P})$ for $\mathcal{P}$ from *Example 1*.

\begin{theorem}
    (Pruesse? and Ruskey) For a poset $\mathcal{P}$, the swap graph $\mathcal{G}(\mathcal{P})$ is connected.
\end{theorem}

That's it? should we move the swap things to motivations?

\section{Motivation}

Given a poset $\mathcal{P}$, finding $|\mathcal{L}(\mathcal{P})|$ is proved by Brightwell and Winkler to be \#P-complete. Pruesse and Ruskey gives a algorithm for generating $\mathcal{L}(\mathcal{P})$ in constant amortized time.

Heath mining poset: generating poset one

Next, we give the problem definition:

\begin{definition}
    (Poset Cover Problem) Given a set of linearizations $\Upsilon$, find a a set of posets $\mathcal{C}$, called a cover, such that $\Upsilon = \bigcup_{P \in C} \mathcal{L}(P)$ and that $|\mathcal{C}|$ is minimal.

    A poset cover $\mathcal{C}$ of a set posets on $S$ for a set of linearizations $\Upsilon$ satisfies the following: (1) (extension constraint) $\forall L \in \Upsilon \exists P \in \mathcal{C}, P \sqsubseteq L$; (2) (non-extension constraint) $\forall L \in \mathfrak{S}(\mathcal{S}) - \Upsilon \forall P \in \mathcal{C}, P \not\sqsubseteq L$.
\end{definition}
The poset cover problem finds, given a set of linearizations $\Upsilon$, a set of posets $\mathcal{C}$, called a cover, such that $\Upsilon = \bigcup_{P \in C} \mathcal{L}(P)$ and that $|\mathcal{C}|$ is minimal. This problem is proved to be NP-complete.(cite)

\begin{theorem}
    insulating barrier method
\end{theorem}

\section{SAT Encoding Part}
We use the boolean logic of z3 SMT solver from MS research. Since poset is connected, by contraposition, we can first divide and conquer on connected components of the swap graph, by solving each component individually.

We then check with incrementing number of posets to find the minimal number of posets to cover each component. We first encode the axioms of posets; namely, reflexivity, antisymmetry, and transitivity. Next, we encode the extension constraint with contraposition from input linearizations to reduce the number of variables. Finally, we encode the non-extension constraint with negation of extension constraint on the insulating barrier.

\section{Exp Part}
NOTE: how do i put graph here?

\section{Conclusions}

\section{References}
NOTE: how to use list?

\section{Appendix}

\begin{theorem}
    Permutations and Nerode
\end{theorem}

\begin{theorem}
    (Corollary of Szpilrajn extension theorem) For a partial order $\leq$, $\leq = \bigcap_{< \in \mathcal{E}(\leq)} <$.
\end{theorem}

\begin{theorem}
    (Heath and Nema) If $x \parallel y$ for $\leq$ and there is $< \in \mathcal{E}(\leq)$ such that $x \prec y$ for $<$, then there is $<' \in \mathcal{E}(\leq)$ such that $y \prec x$ for $<'$.
\end{theorem}

\begin{theorem}
    Swap graphs are connected by Kendall tau paths
\end{theorem}

\end{document}
