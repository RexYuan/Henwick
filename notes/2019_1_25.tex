\documentclass[12pt]{llncs}

\usepackage[utf8]{inputenc}
\usepackage[T1]{fontenc}

\usepackage{CJKutf8}
\usepackage{graphicx}
\usepackage{amsmath}
\usepackage{amssymb}
\usepackage{mathtools}
\usepackage{hyperref}
\usepackage{color}
\usepackage[most]{tcolorbox}
\usepackage{array}
\usepackage{enumitem}
\usepackage{centernot}
\usepackage{extsizes}
\usepackage{datetime}
\usepackage{mathabx}
\usepackage{tikz}
\usepackage[linesnumbered,lined,boxed,commentsnumbered,vlined]{algorithm2e}
\usepackage{parskip}
\usepackage{minted}
\usepackage[varqu]{zi4}

\usepackage{bussproofs}

\newdate{date}{25}{1}{2019}

\DeclarePairedDelimiter\set\{\}
\DeclarePairedDelimiter\tuple()
\DeclarePairedDelimiter\paren()
\DeclarePairedDelimiter{\abs}{\lvert}{\rvert}

\renewcommand{\implies}{\rightarrow}

\newcommand{\bicond}{\leftrightarrow}
\newcommand{\smodels}{\vdash}
\newcommand{\byrule}[1]{\RightLabel{\textit{#1}}}
\newcommand{\bye}[1]{}
\newcommand{\LeftToRight}{$\Longrightarrow$}
\newcommand{\RightToLeft}{$\Longleftarrow$}
\newcommand{\T}{{\textbf T}}
\newcommand{\F}{{\textbf F}}
\newcommand{\True}{{\textbf{True}}}
\newcommand{\False}{{\textbf{False}}}
\newcommand{\struct}[1]{\mathcal{#1}}
\newcommand{\arity}[1]{\mathrm{ar}(#1)}
\newcommand{\rank}[1]{\mathrm{qr}(#1)}
\newcommand{\maxx}[1]{\mathrm{max}(#1)}
\newcommand{\free}[1]{\mathrm{free}(#1)}
\newcommand{\var}[1]{\mathrm{var}(#1)}
\newcommand{\eqq}{\approx}
\newcommand{\subst}[2]{[#1/#2]}
\newcommand{\val}{\mathbf{val}}
\newcommand{\modelof}[1]{\mathrm{Md}(#1)}
\newcommand{\consequenceof}[1]{\mathrm{Cn}(#1)}
\newcommand{\theoryof}[1]{\mathrm{Th}(#1)}
\newcommand{\isomorph}{\cong}
\newcommand{\FO}{\mathrm{FO}}
\newcommand{\satof}[1]{\mathsf{SAT}(#1)}
\newcommand{\finsatof}[1]{\textsf{FIN-SAT}(#1)}
\newcommand{\validityof}[1]{\textsf{Validity}(#1)}
\newcommand{\finvalidityof}[1]{\textsf{Finite-Validity}(#1)}
\newcommand{\Robinson}{\mathsf{Q}}
\newcommand{\Successor}{\mathsf{Succ}}
\newcommand{\Proj}[1]{\Pi_{#1}}
\newcommand{\Select}[1]{\sigma_{#1}}
\newcommand{\LJoin}{\ltimes}
\newcommand{\kA}{k\text{A}}

\newcommand{\sA}{\struct{A}}
\newcommand{\sB}{\struct{B}}
\newcommand{\sN}{\struct{N}}
\newcommand{\sM}{\struct{M}}

\newcommand{\cC}{\mathcal{C}}
\newcommand{\cD}{\mathcal{D}}
\newcommand{\cF}{\mathcal{F}}
\newcommand{\cW}{\mathcal{W}}
\newcommand{\cQ}{\mathcal{Q}}
\newcommand{\cR}{\mathcal{R}}

\newcommand\aug{\fboxsep=-\fboxrule\!\!\!\fbox{\strut}\!\!\!}

\newcolumntype{M}[1]{>{\centering\arraybackslash}m{#1}}

\begin{document}
\displaydate{date}

\section{SQL Stuff}
\begin{definition}
Let $\cW = \tuple{\cC \cup \bigtimes\cF \cup (\bigtimes\cF)^{2}, \eqq, \leq, \Pi, \sigma, \ltimes, 1, 2, 3, ...}$ where
\begin{itemize}
    \item $\cC$ is a set of domain names
    \item $\cF$ is a set of domain sets
    \item $\eqq$ is a relation that compares entries
    \item $\leq$ is a relation that compares cardinalities
    \item $\Pi, \sigma, \ltimes$ are usual relational algebraic functions
    \item $1, 2, 3, ...$ are constant sets with cardinalities $1, 2, 3, ...$
\end{itemize}
\end{definition}
A database $\cD \in (\bigtimes\cF)^{2}$ is an embedding in $\cW$.

\begin{definition}[$k$-Anonymity]
For $k \in \set{1,2,3,...}$ and $\cQ \subseteq \cC$, a database $\cD$ is $k$-anonymous in regards to $\cQ$ iff it holds that
\[
\forall_{d \in \cD}
\exists_{S \subseteq \cD}\ \
(
S \geq k \wedge
\forall_{s \in S}\ \
\Proj{\cQ} s \eqq \Proj{\cQ} d
)
\]
\end{definition}
A SQL query to discover all the values that violate $k$-Anonymity is
\begin{minted}[frame=single,breaklines,autogobble,fontfamily=zi4,escapeinside=||,mathescape=true]{sql}
SELECT |$\cQ$|, COUNT(*) AS count FROM |$\cD$| GROUP BY |$\cQ$| HAVING count < |$k$|
\end{minted}

\begin{definition}[$\ell$-Diversity]
For $l \in \set{1,2,3,...}$ and $\cR \subseteq \cC - \cQ$, a database $\cD$ that's $k$-anonymous in regards to $\cQ$ is $\ell$-diverse in regards to $\cR$ iff it holds that
\[
\forall_{d \in \cD}
\exists_{S \subseteq \cD}\ \
(
S \geq k \wedge
\forall_{s \in S}\ \
\Proj{\cQ} s \eqq \Proj{\cQ} d
\wedge
\Proj{\cR} S \geq \ell
)
\]
\end{definition}
A SQL query to discover all the values that violate $\ell$-Diversity is
\begin{minted}[frame=single,breaklines,autogobble,fontfamily=zi4,escapeinside=||,mathescape=true]{sql}
SELECT |$\cQ$|, COUNT(*) AS count FROM
    (SELECT |$\cQ$| FROM |$\cD$| GROUP BY |$\cQ$,$\cR$|) AS T
GROUP BY |$\cQ$| HAVING count < |$\ell$|
\end{minted}

\begin{definition}[$(X,Y)$-Privacy]
For $k \in \set{1,2,3,...}$ and $X,Y \subseteq \cC$, a database is $(X,Y)$-anonymous for $k$ iff it holds that
\[
\forall_{d \in \cD}
\exists_{S \subseteq \cD}\
(
\Proj{Y} S \geq k \wedge
\forall_{s \in S}\
\Proj{X} s \eqq \Proj{X} d
)
\]
\end{definition}

\begin{minted}[frame=single,breaklines,autogobble,fontfamily=zi4,escapeinside=||,mathescape=true]{sql}
SELECT |$X$|, count(*) AS count FROM
    (SELECT |$X$|, count(*) FROM |$\cD$| GROUP BY |$X,Y$|)
AS T GROUP BY |$X$| HAVING count < |$k$|
\end{minted}

\end{document}
