\documentclass[12pt]{llncs}

\usepackage{rex}

\usepackage{stmaryrd}
\usepackage[retainorgcmds]{IEEEtrantools}

% alternative caligraphy
\usepackage{calrsfs}
\DeclareMathAlphabet{\pazocal}{OMS}{zplm}{m}{n}

% alternative caligraphy
\DeclareFontFamily{U}{mathc}{}
\DeclareFontShape{U}{mathc}{m}{it}%
{<->s*[1.03] mathc10}{}
\DeclareMathAlphabet{\mathscr}{U}{mathc}{m}{it}

% minted python shortcut
\newminted{py}{frame=single,breaklines,autogobble,fontfamily=zi4,
escapeinside=||,mathescape=true,style=bw,linenos,python3=true}
\newmintedfile[pyfile]{py}{frame=single,breaklines,autogobble,fontfamily=zi4,
escapeinside=||,mathescape=true,linenos,python3=true}

\spnewtheorem{thing}{Thing}{\bfseries}{\itshape}

\newcommand{\cM}{\mathcal{M}}
\newcommand{\cT}{\mathcal{T}}
\newcommand{\MDNF}{\mathrm{M}_\mathrm{DNF}}
\newcommand{\MTERM}{\mathrm{M}_\mathrm{TERM}}

\begin{document}
\section*{Lambda Algorithm}
For $f$
\begin{IEEEeqnarray*}{rCl}
f & = & \bigvee_{n=1}^{s} \bigwedge_{m}^{t_n} \ell^n_m =
\bigvee_{n=1}^{s} T_n \\
f(x + a_k) & = & \bigvee_{n=1}^{s}
( (\bigwedge_{\substack{x_i \in P\ell^n \\ a_k[i] = 0}} x_i) \wedge
  (\bigwedge_{\substack{x_i \in P\ell^n \\ a_k[i] = 1}} \neg x_i) \wedge
  (\bigwedge_{\substack{x_j \in N\ell^n \\ a_k[j] = 0}} \neg x_j) \wedge
  (\bigwedge_{\substack{x_j \in N\ell^n \\ a_k[j] = 1}} x_j)
) \\
& = & \bigvee_{n=1}^{s} T_n^{a_k} \\
\cM( f(x + a_k) ) & = & \bigvee_{n=1}^{s}
( (\bigwedge_{\substack{x_i \in P\ell^n \\ a_k[i] = 0}} x_i) \wedge
  (\bigwedge_{\substack{x_j \in N\ell^n \\ a_k[j] = 1}} x_j)
) \\
& = & \bigvee_{n=1}^{s} \set{
    \bigwedge_{(x + a_k)[i] = 1} x_i \mid
    x \in \ell^n
} \\
& = & \bigvee_{n=1}^{s} \cT_k =
\bigvee_{n=1}^{s} \cM(T_n^{a_k}) \\
\cM_{a_k} ( f ) & = & \cM( f(x + a_k) )(x + a_k)
= (\bigvee_{n=1}^{s} \cT_k)(x + a_k)
\end{IEEEeqnarray*}

For $S_i = \set{\widehat{v_n} + a_i \mid n}$
\begin{IEEEeqnarray*}{rCl}
\MDNF(S_i) & = & \set{
    \MTERM(\widehat{v_n} + a_i)) \mid
    (\widehat{v_n} + a_i) \in S_i
} \\
& = & \set{
    \bigwedge_{(\widehat{v_n} + a_i)[j] = 1} x_j \mid
    (\widehat{v_n} + a_i) \in S_i
} \\
\bigvee_{(\widehat{v_n} + a_i) \in S_i} \MDNF(S_i) & = &
  \bigvee_{(\widehat{v_n} + a_i) \in S_i}
    \bigwedge_{(\widehat{v_n} + a_i)[j] = 1} x_j \\
H_i & = &
(\bigvee_{(\widehat{v_n} + a_i) \in S_i} \MDNF(S_i)) (x + a_i) \\
& = & \bigvee_{(\widehat{v_n} + a_i) \in S_i}(
  (\bigwedge_{\substack{\widehat{v_n}[j] = 1 \\ a_i[j] = 0}} x_j)
  \wedge
  (\bigwedge_{\substack{\widehat{v_n}[j] = 0 \\ a_i[j] = 1}} \neg x_j)
)
\end{IEEEeqnarray*}

The idea is to fit $\MDNF(S_i)$ to $\cT_i$
in each $a_i$ dimension s.t.
$H_i$ fits to $\cM_{a_k} ( f )$.

\newpage
\subsubsection*{Proof}
Assume
\begin{enumerate}
\item $v^\delta$ is a positive counterexample.
\item $I^\delta$ is non-empty.
\item $\MTERM(\widehat{v}^\delta + a_i) \in \cT_i - \MDNF(S_i^{\delta-1})$ for $i \in I^\delta$.
\item $\MDNF(S_i^\delta) \subseteq \cT_i$ for $i = 1 .. t$.
\item $H_i^\delta \to \cM_{a_i}(f)$ for $i = 1 .. t$.
\end{enumerate}

Then induction is as follows.
\begin{enumerate}[a.]
\item By hypothesis 5,
\begin{IEEEeqnarray*}{rCl}
H_i^\delta & \to & \cM_{a_i}(f) \\
\bigwedge_{i=1}^{t} H_i^\delta & \to & \bigwedge_{i=1}^{t} \cM_{a_i}(f) = f
\end{IEEEeqnarray*}
Therefore $v^{\delta+1}$ is a positive counterexample.

\item Since (a), $v^{\delta+1}$ is a positive counterexample,
$(\bigwedge_{i=1}^{t} H_i^\delta) (v^{\delta+1}) = 0$
and so
\[
I^{\delta+1} = \set{i \mid H_i^\delta(v^{\delta+1}) = 0}
\]
is non-empty.

\item Let $i \in I^{\delta+1}$.
Since (b), $H_i^\delta (v^{\delta+1}) = 0$.
By definition,
\[
H_i^\delta (v^{\delta+1}) = 
(\bigvee_{(\widehat{v_n} + a_i) \in S_i^\delta} \MDNF(S_i^\delta)) (v^{\delta+1} + a_i) = 0
\]
After walking, $v^{\delta+1}$ becomes $\widehat{v}^{\delta+1}$ and
$\widehat{v}^{\delta+1} \leq_{a_i} v^{\delta+1}$ and
\[
\widehat{v}^{\delta+1} + a_i \leq v^{\delta+1} + a_i
\]
Since $(\bigvee_{(\widehat{v_n} + a_i) \in S_i^\delta} \MDNF(S_i^\delta))$
is monotone,
\[
(\bigvee_{(\widehat{v_n} + a_i) \in S_i^\delta} \MDNF(S_i^\delta)) (\widehat{v}^{\delta+1} + a_i) = 0
\]
so
\[
\MTERM(\widehat{v}^{\delta+1} + a_i) \notin \MDNF(S_i^\delta)
\]
Since $\widehat{v}^{\delta+1}$ is walked towards $a_i$,
\[
f(\widehat{v}^{\delta+1}) =
f((\widehat{v}^{\delta+1} + a_i) + a_i) = 1
\]
and for all $(\widehat{v}^{\delta+1} + a_i)[j] = 1$
\[
f(\widehat{v}^{\delta+1} + \neg x_j) =
f(\widehat{v}^{\delta+1} + \neg x_j + a_i + a_i) = 
f((\widehat{v}^{\delta+1} + a_i) + \neg x_j + a_i) = 0
\]
So by Proposition A, there is a term $T$ in any DNF of
$f(x + a_i)$ such that $\cM(T) = \MDNF(\widehat{v}^{\delta+1} + a_i)$.
Therefore
\[
\MDNF(\widehat{v}^{\delta+1} + a_i) \in \cT_i = \set{
    \cM(T) \mid T \in f(x + a_i)
}
\]
This is the crux of the proof.

\item If $i \notin I^{\delta+1}$, then
$S_i^{\delta+1} = S_i^{\delta}$. If $i \in I^{\delta+1}$,
then
\[
S_i^{\delta+1} = S_i^{\delta} \cup
\set{\widehat{v}^{\delta+1} + a_i}.
\]
Either way, by hypothesis 4 and (d), we have
$\MDNF(S_i^{\delta+1}) \subseteq \cT_i$.

\item By definition,
\[
H_i^{\delta+1} (x) =
(\bigvee_{(\widehat{v_n} + a_i) \in S_i^{\delta+1}}
\MDNF(S_i^{\delta+1}))
(x + a_i)
\]
and
\[
\cM_{a_k} ( f ) = (\bigvee_{n=1}^{s} \cT_k)(x + a_k)
\]
By (d), $\MDNF(S_i^{\delta+1}) \subseteq \cT_i$, so
\[
\bigvee_{(\widehat{v_n} + a_i) \in S_i^{\delta+1}}
\MDNF(S_i^{\delta+1}) \to
\bigvee_{n=1}^{s} \cT_k
\]
Therefore
\[
H_i^{\delta+1} (x) \to \cM_{a_k} ( f )
\]


\end{enumerate}





\newpage
%\pyfile{../bshouty/cdnf.py}

\end{document}
