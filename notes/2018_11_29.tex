\documentclass[12pt]{llncs}

\usepackage[utf8]{inputenc}
\usepackage[T1]{fontenc}

\usepackage{CJKutf8}
\usepackage{graphicx}
\usepackage{amsmath}
\usepackage{amssymb}
\usepackage{mathtools}
\usepackage{hyperref}
\usepackage{color}
\usepackage[most]{tcolorbox}
\usepackage{array}
\usepackage{enumitem}
\usepackage{centernot}
\usepackage{extsizes}
\usepackage{datetime}
\usepackage{mathabx}
\usepackage{tikz}
\usepackage[linesnumbered,lined,boxed,commentsnumbered,vlined]{algorithm2e}
\usepackage{parskip}

\usepackage{bussproofs}

\newdate{date}{29}{11}{2018}

\DeclarePairedDelimiter\set\{\}
\DeclarePairedDelimiter\tuple()
\DeclarePairedDelimiter\paren()
\DeclarePairedDelimiter{\abs}{\lvert}{\rvert}

\renewcommand{\implies}{\rightarrow}

\newcommand{\bicond}{\leftrightarrow}
\newcommand{\smodels}{\vdash}
\newcommand{\byrule}[1]{\RightLabel{\textit{#1}}}
\newcommand{\bye}[1]{}
\newcommand{\LeftToRight}{$\Longrightarrow$}
\newcommand{\RightToLeft}{$\Longleftarrow$}
\newcommand{\T}{{\textbf T}}
\newcommand{\F}{{\textbf F}}
\newcommand{\True}{{\textbf{True}}}
\newcommand{\False}{{\textbf{False}}}
\newcommand{\struct}[1]{\mathcal{#1}}
\newcommand{\arity}[1]{\mathrm{ar}(#1)}
\newcommand{\rank}[1]{\mathrm{qr}(#1)}
\newcommand{\maxx}[1]{\mathrm{max}(#1)}
\newcommand{\free}[1]{\mathrm{free}(#1)}
\newcommand{\var}[1]{\mathrm{var}(#1)}
\newcommand{\eqq}{\approx}
\newcommand{\subst}[2]{[#1/#2]}
\newcommand{\val}{\mathbf{val}}
\newcommand{\modelof}[1]{\mathrm{Md}(#1)}
\newcommand{\consequenceof}[1]{\mathrm{Cn}(#1)}
\newcommand{\theoryof}[1]{\mathrm{Th}(#1)}
\newcommand{\isomorph}{\cong}
\newcommand{\FO}{\mathrm{FO}}
\newcommand{\satof}[1]{\mathsf{SAT}(#1)}
\newcommand{\finsatof}[1]{\textsf{FIN-SAT}(#1)}
\newcommand{\validityof}[1]{\textsf{Validity}(#1)}
\newcommand{\finvalidityof}[1]{\textsf{Finite-Validity}(#1)}
\newcommand{\Robinson}{\mathsf{Q}}
\newcommand{\Successor}{\mathsf{Succ}}
\newcommand{\Proj}[1]{\Pi_{#1}}

\newcommand{\sA}{\struct{A}}
\newcommand{\sB}{\struct{B}}
\newcommand{\sN}{\struct{N}}
\newcommand{\sM}{\struct{M}}

\newcommand{\cC}{\mathcal{C}}
\newcommand{\cD}{\mathcal{D}}
\newcommand{\cF}{\mathcal{F}}

\newcommand\aug{\fboxsep=-\fboxrule\!\!\!\fbox{\strut}\!\!\!}

\newcolumntype{M}[1]{>{\centering\arraybackslash}m{#1}}

\begin{document}
\displaydate{date}

\section{Preliminaries}
\begin{definition}
A database $\cD$ is a finite multi-set of $m$-tuples associated with an $m$-tuple for column names $\cC$ and an $m$-tuple of finite sets for column domains $\cF$, such that $\cD \subseteq \bigtimes_{F \in \cF} F$. Let the function of taking the absolute be such that it implicitly transforms a multi-set into a set by removing the duplicates.
\end{definition}

\begin{definition}[$k$-Anonymity]
$k$ is an integer. A database $\cD$ is $k$-anonymous iff, for all entries $d \in \cD$ and some subsets of column names $Q \subseteq \cC$, there exist a subset $S \subseteq \cD$ of size at least $k$ such that $\Proj{Q} s = \Proj{Q} d$ for all $s \in S$.
\end{definition}

\begin{definition}[MultiR $k$-Anonymity]
$k$-Anonymity with multiple databases joined.
\end{definition}

\begin{definition}[$\ell$-Diversity]
Extension of $k$-Anonymity. $\ell$ is an integer. A database $\cD$ is $\ell$-diverse iff, for all entries $d \in \cD$ and for all subsets of column names $Q \subseteq \cC$, there exist a subset $S \subseteq \cD$ of size at least $k$ such that $\Proj{Q} s = \Proj{Q} d$ for all $s \in S$, and for some other columns $Q' \subseteq \cC - Q$, $\abs{\Proj{c} S}$ is at least $\ell$ for all $c \in Q'$.
\end{definition}

\begin{definition}[Confidence Bounding]
Probabilistic property.
\end{definition}

\begin{definition}[$(\alpha, k)$-Anonymity]
\end{definition}

\begin{definition}[$(X,Y)$-Privacy]
$X,Y \subseteq C$ are subsets of column names. A database $\cD$ is $(X,Y)$-anonymous for some integer $k$ iff, for all entries $d \in \cD$ and all columns $x \in X$, there is $S \subseteq \cD$ such that $\Proj{x} d = \Proj{x} s$ for all $s \in S$, and $\abs{\Proj{y} S}$ is at least $k$ for all $y \in Y$.
\end{definition}

\begin{definition}[$(k, e)$-Anonymity]
\end{definition}

\begin{definition}[$(\epsilon,m)$-Anonymity]
\end{definition}

\begin{definition}[Personalized Privacy]
\end{definition}

\begin{definition}[$t$-closeness]
\end{definition}

\begin{definition}[$\delta$-Presence]
\end{definition}

\begin{definition}[$(c,t)$-Isolation]
\end{definition}

\begin{definition}[$\epsilon$-Differential Privacy]
\end{definition}

\begin{definition}[$(d, \gamma)$-Privacy]
\end{definition}

\begin{definition}[Distributional Privacy]
\end{definition}

\end{document}
