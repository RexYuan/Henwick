\documentclass[12pt]{llncs}

\usepackage{rex}

\usepackage{stmaryrd}
\usepackage[retainorgcmds]{IEEEtrantools}

\DeclareMathOperator{\probability}{\mathsf{Pr}}
\newcommand{\pr}[1]{\probability(#1)}
\DeclareMathOperator{\reward}{\mathsf{R}}
\newcommand{\rw}[1]{\reward(#1)}
\DeclareMathOperator{\probing}{\mathbb{P}}
\newcommand{\pb}[1]{\probing \llbracket #1 \rrbracket}
\DeclareMathOperator{\expectedp}{\mathbb{EP}}
\newcommand{\ep}[1]{\expectedp \llbracket #1 \rrbracket}
\DeclareMathOperator{\weakp}{\mathsf{wp}}
\renewcommand{\wp}[2]{\weakp[#1](#2)}
\DeclareMathOperator{\expectedr}{\mathbb{ER}}
\newcommand{\er}[1]{\expectedr \llbracket #1 \rrbracket}
\DeclareMathOperator{\lexpectedr}{\mathbb{LER}}
\newcommand{\ler}[1]{\lexpectedr \llbracket #1 \rrbracket}
\DeclareMathOperator{\sink}{\mathtt{sink}}
\DeclareMathOperator{\init}{\mathtt{init}}
\DeclareMathOperator{\future}{\mathsf{F}}
\renewcommand{\F}[1]{\future#1}

\newdate{date}{3}{5}{2019}

\begin{document}
\displaydate{date}

% NOTE::::: More general than theirs!
% (1) normal case
% (2) liberal case ER + PROB of divergence

\begin{definition}
Let $S$ be the program states in model $\sM$.
A path is a sequence $\pi = s_1 s_2 s_3 ...$ permitted by $\sM$.
A variable $v$ a function $v : S \mapsto \Reals$.
Every non-empty finite path $\pi = s_1 ... s_i$ determines a
probe state $p$ such that we take $v(p) = v(s_i)$ for any $v$.
Henceforth a path is taken to contain a mix of normal program
states and probe states.
\end{definition}

\begin{definition}
The \emph{probed value} $\pb{v \mid \pi}$ of variable $v$
regarding path $\pi$ is:
\begin{itemize}
\item if $\pi$ is finite,
    \begin{itemize}
    \item $v(p)$ where $p$ is the last probe state in $\pi$ and
    \item undefined if there is no probe state in $\pi$;
    \end{itemize}
\item and, if $\pi$ is infinite,
    \begin{itemize}
    \item something %TODO
    \item undefined if there is no probe state in $\pi$;
    \end{itemize}
\end{itemize}
\end{definition}

\begin{definition}
The \emph{expected probed value} $\ep{v \mid T}$ of variable $v$
regarding paths $T$ is:
\[
\sum_{\pi \in T} \pr{\pi} \pb{v \mid \pi}
\]
\end{definition}

For a program $P$ without probe states and variable $v$, insert probe states $P_{\top}$ immediately after $\init$ and $P_{\bot}$ just before $\sink$:
\begin{IEEEeqnarray*}{rCl}
\er{\F{\sink} \mid P^{v}}
& = & \sum_{\pi \in \F{\sink}} \pr{\pi} \rw{\pi} \\
& = & \sum_{\pi \in \F{\sink}} \pr{\pi} \pb{v \mid \pi}
= \ep{v \mid \F{\sink}}
\end{IEEEeqnarray*}
and
\begin{IEEEeqnarray*}{rCl}
\ler{\F{\sink} \mid P^{v}}
& = & \er{\F{\sink} \mid P^{v}} + \pr{\neg \F{\sink} \mid P^{v}} \\
& = & \er{\F{\sink} \mid P^{v}} + \sum_{\pi \in \neg \F{\sink}} \pr{\pi} \\
& = & \er{\F{\sink} \mid P^{v}} + \sum_{\pi \in \neg\F{\sink}} \pr{\pi} \pb{1 \mid \pi} \\
& = & \ep{v \mid \F{\sink}} + \ep{1 \mid \neg \F{\sink}}
\end{IEEEeqnarray*}

\end{document}
