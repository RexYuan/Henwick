\documentclass[12pt]{llncs}

\usepackage{rex}

\usepackage{stmaryrd}
\usepackage[retainorgcmds]{IEEEtrantools}

% alternative caligraphy
\usepackage{calrsfs}
\DeclareMathAlphabet{\pazocal}{OMS}{zplm}{m}{n}

% alternative caligraphy
\DeclareFontFamily{U}{mathc}{}
\DeclareFontShape{U}{mathc}{m}{it}%
{<->s*[1.03] mathc10}{}
\DeclareMathAlphabet{\mathscr}{U}{mathc}{m}{it}

% minted python shortcut
\newminted{py}{frame=single,breaklines,autogobble,fontfamily=zi4,
escapeinside=||,mathescape=true,style=bw,linenos,python3=true}

\DeclareMathOperator{\probability}{\mathsf{Pr}}
\newcommand{\pr}[1]{\probability(#1)}
\DeclareMathOperator{\reward}{\mathsf{R}}
\newcommand{\rw}[1]{\reward(#1)}
\DeclareMathOperator{\probing}{\mathbb{P}}
\newcommand{\pb}[1]{\probing \llbracket #1 \rrbracket}
\DeclareMathOperator{\expectedp}{\mathbb{EP}}
\newcommand{\ep}[1]{\expectedp \llbracket #1 \rrbracket}
\DeclareMathOperator{\weakp}{\mathsf{wp}}
\renewcommand{\wp}[2]{\weakp[#1](#2)}
\DeclareMathOperator{\expectedr}{\mathbb{ER}}
\newcommand{\er}[1]{\expectedr \llbracket #1 \rrbracket}
\DeclareMathOperator{\lexpectedr}{\mathbb{LER}}
\newcommand{\ler}[1]{\lexpectedr \llbracket #1 \rrbracket}
\DeclareMathOperator{\sink}{\mathtt{sink}}
\DeclareMathOperator{\init}{\mathtt{init}}
\DeclareMathOperator{\future}{\mathsf{F}}
\renewcommand{\F}[1]{\future#1}
\newcommand{\PROBE}{\mathscr{Probe}}
\newcommand{\pchoice}[1]{\mathbin{[#1]}}

\newdate{date}{3}{5}{2019}

\begin{document}
\displaydate{date}

% NOTE::::: More general than theirs!
% (1) normal case
% (2) liberal case ER + PROB of divergence

\begin{definition}
Let $S$ be the program states in program $P$.
A path is a sequence $\pi = s_1 s_2 s_3 ...$ permitted by $P$.
A variable $v$ a function $v : S \mapsto \Reals$.
Every non-empty finite path $\pi = s_1 ... s_i$ determines a
probe state $p_{\pi}$ such that we take $v(p_{\pi}) = v(s_i)$ for
any $v$.
Henceforth a path is taken to contain a mix of normal program
states and probe states. Write $p$ instead of $p_{\pi}$ since
there can be no confusion.
For example, let
\[
\pi =
s_1 \cdot p_{s_1} \cdot s_2 \cdot s_3 \cdot
p_{s_1 p_{s_1} s_2 s_3} \cdot
p_{s_1 p_{s_1} s_2 s_3 p_{s_1 p_{s_1} s_2 s_3}}
\]
write
\[
\pi = s_1 \cdot p \cdot s_2 \cdot s_3 \cdot p \cdot p
\]
\end{definition}

\begin{definition}
The \emph{probed value} $\pb{v \mid \pi}$ of variable $v$
regarding path $\pi$ is:
\begin{itemize}
\item $v(p)$ where $p$ is the last(?) probe state in $\pi$ and
\item undefined if there is no probe state in $\pi$.
\end{itemize}
\end{definition}

\begin{definition}
The \emph{expected probed value} $\ep{v \mid \Pi}$ of variable $v$
regarding paths $\Pi$ is:
\[
\sum_{\pi \in \Pi} \pr{\pi} \pb{v \mid \pi}
\]
\end{definition}

\section*{Generality}

For a program $P$ without probe states and variable $v$, insert probe states immediately after $\init$ and just before $\sink$:
\begin{IEEEeqnarray*}{rCl}
\er{\F{\sink} \mid P^{v}}
& = & \sum_{\pi \in \F{\sink}} \pr{\pi} \rw{\pi} \\
& = & \sum_{\pi \in \F{\sink}} \pr{\pi} \pb{v \mid \pi}
= \ep{v \mid \F{\sink}}
\end{IEEEeqnarray*}
and
\begin{IEEEeqnarray*}{rCl}
\ler{\F{\sink} \mid P^{v}}
& = & \er{\F{\sink} \mid P^{v}} + \pr{\neg \F{\sink} \mid P^{v}} \\
& = & \er{\F{\sink} \mid P^{v}} + \sum_{\pi \in \neg \F{\sink}} \pr{\pi} \\
& = & \er{\F{\sink} \mid P^{v}} + \sum_{\pi \in \neg\F{\sink}} \pr{\pi} \pb{1 \mid \pi} \\
& = & \ep{v \mid \F{\sink}} + \ep{1 \mid \neg \F{\sink}}
\end{IEEEeqnarray*}

\pagebreak

What should $\ep{x \mid \top}$ be for:

\begin{listing}[H]
\begin{pycode}
x = 1
|$\PROBE$|
x = 2
|$\PROBE$|
x = 3
\end{pycode}
\caption{We'd want $\ep{x \mid \top} = 3$ and that compels the
definition of $\probing$ to take the \emph{last} probe state.}
\end{listing}

\begin{listing}[H]
\begin{pycode}
while true:
    x = 1
    |$\PROBE$|
\end{pycode}
\caption{We'd want $\ep{x \mid \top} = 1$ but the probe state occurs
infinitely often so there is no \emph{last} probe state.}
\end{listing}

\begin{listing}[H]
\begin{pycode}
while true:
    x = 1 |$\pchoice{0.5}$| x = 2
    |$\PROBE$|
\end{pycode}
\caption{We'd want $\ep{x \mid \top} = 0.5$ and even if we take
not the last probe state but every probed value weighted by its
path's probability, we would count in too much:
$0.5 \cdot 1 + 0.5 \cdot 2 + 0.25 \cdot 1 + 0.25 \cdot 2
+ 0.25 \cdot 1 + 0.25 \cdot 2 + ...$.}
\end{listing}

\begin{listing}[H]
\begin{pycode}
x = 1
while true:
    x = x |$\cdot$| 0.5
    |$\PROBE$|
\end{pycode}
\caption{We'd want $\ep{x \mid \top} = 0$ but I have no idea
how would this work since $\sum_{i} 0.5^i = \frac{0.5}{1-0.5} = 1$.
For these two I don't think I know enough to properly handle
them in a meaningful way. I'm still reading on semantics.}
\end{listing}

\end{document}
